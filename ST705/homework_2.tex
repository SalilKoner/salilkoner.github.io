\documentclass[11pt]{article}
\usepackage{amsmath,amssymb,color,float}
\usepackage{graphicx,psfrag,epsf}
\usepackage{natbib}


\setlength{\oddsidemargin}{.15in} 
\setlength{\textwidth}{6.25in}
\setlength{\topmargin}{-0.25in}
\setlength{\headheight}{-0.15in}
\setlength{\textheight}{8.9in} 

\linespread{1.25}


\title{ST 705 Linear models and variance components \\ 
        Homework problem set 2}


\begin{document}
\maketitle

\begin{enumerate}

\item Let $A$ be a positive definite matrix, and show that 
\[
\text{tr}(I - A^{-1}) \le \log\det(A) \le \text{tr}(A - I).
\]

\item Show that the covariance function defined for $X, Y \in \mathbb{R}^{p}$ by
\[
\text{Cov}(X,Y) := E[(X - E(X))(Y - E(Y))']
\]
satisfies the following properties.  For random variables $X, Y, Z \in \mathbb{R}^{p}$ with finite covariance, and any $c \in \mathbb{R}$,
\begin{enumerate}
\item $\text{Cov}(X+Y,Z) = \text{Cov}(X,Z) + \text{Cov}(Y,Z)$
\item $\text{Cov}(cX,Y) = c\cdot \text{Cov}(X,Y)$
\item $\text{Cov}(X,Y)^{*} = \text{Cov}(Y,X)$
\item $\text{Cov}(X,X) \ge 0$ for all $X$, and $\text{Cov}(X,X) = 0$ implies that $X$ is constant a.s.
\end{enumerate}
Then, deduce that if $p = 1$ the covariance is an inner product over some (quotient) vector space, and if $p > 1$ the the function $f(X,Y) := \text{tr}\big(\text{Cov}(X,Y)\big)$ is an inner product.

\item Exercise A.50 from Monahan.

\item For matrices $A \in \mathbb{R}^{p\times q}$, the \textit{spectral} norm is defined as,
\[
\|A\|_{2} := \sqrt{\sup_{x\ne0}\frac{x'A'Ax}{x'x}}.
\]
Further, the eigenvalues of $A'A$ are the squares of the \textit{singular values} of $A$, so sometimes the definition of the spectral norm is expressed as
 \[
\|A\|_{2} := \sigma_{\max}(A),
\]
where $\sigma_{\max}$ denotes the largest singular value of $A$. 
\begin{enumerate}
\item Verify that the spectral norm is a norm.  Recall that a norm must satisfy the following axioms for any $A,B,C \in \mathbb{R}^{p\times q}$ and any $\alpha \in \mathbb{R}$.
\begin{enumerate}
\item $\|\alpha A\| = |\alpha|\|A\|$
\item $\|A + B\| \le \|A\| + \|B\|$
\item $\|A\| \ge 0$ with equality if and only if $A = 0$.
\end{enumerate}
\item Show that the spectral norm is sub-multiplicative for square matrices.  That is, for $A,B \in \mathbb{R}^{p\times p}$, $\|AB\|_{2} \le \|A\|_{2}\|B\|_{2}$.  
\end{enumerate}

\item Suppose you do not know that the rank of a matrix is equal to the number of nonzero singular values.  Show that the rank of a projection matrix is equal to its trace.  First think about how to show this in the symmetric case, and then consider the more general case of a non-symmetric idempotent matrix.

\item Show that if $\text{rank}(BC) = \text{rank}(B)$, then $\text{col}(BC) = \text{col}(B)$, where col$(\cdot)$ denotes the column space.








\end{enumerate}






\end{document}