\documentclass[11pt]{article}
\usepackage{amsmath,amssymb,color,float}
\usepackage{graphicx,psfrag,epsf}
\usepackage{natbib}


\setlength{\oddsidemargin}{.15in} 
\setlength{\textwidth}{6.25in}
\setlength{\topmargin}{-0.25in}
\setlength{\headheight}{-0.15in}
\setlength{\textheight}{8.9in} 

\linespread{1.25}

\newtheorem{theorem}{Theorem}
\newtheorem{definition}[theorem]{Definition}



\title{ST 705 Linear models and variance components \\ 
        Homework problem set 9}


\begin{document}
\maketitle

\begin{enumerate}

\item(2 points) Exercise 4.15 from Monahan.

\item(2 points) Exercise 4.21 from Monahan.

\item(2 points) Exercise 4.22 from Monahan.

\item(2 points) Exercise 4.23 from Monahan.

\item(2 points) Exercise 5.2 from Monahan.

\item(2 points) Recall the definition of the multivariate normal distribution from class:

\begin{definition}
The $p$-dimensional random vector $Y$ is said to follow the multivariate normal distribution with mean $\mu$ and covariance matrix $\Sigma$ if for every $p$-dimensional vector $v$ such that $v'\Sigma v \ne 0$,
\[
v'Y \sim \text{N}(v'\mu,v'\Sigma v).
\]
Denote $Y \sim \text{N}_{p}(\mu,\Sigma)$. $\hfill \blacksquare$
\end{definition}
Prove that if $\Sigma$ is nonsingular, then $Y \sim \text{N}_{p}(\mu,\Sigma)$ if and only if $Y$ has density,
\[
f(y) = \det(2\pi\Sigma)^{-\frac{1}{2}}e^{-\frac{1}{2}(y-\mu)'\Sigma^{-1}(y-\mu)}.
\]

\item(2 points) Construct two random variables that have zero correlation, but are {\em not} independent.

\item(2 points) Exercise 5.6 from Monahan.
\end{enumerate}

{\noindent\bf Optional lab problems}

\begin{enumerate}

\item Let $A$ be an $n\times n$ matrix.  Show that if $A$ is positive-definite, then it must be symmetric.  Construct a counter example if this statement is not true.  Do not simply appeal to the Cholesky factorization.

\item What is the contrapositive of the statement given in the previous problem?  Think about what this contrapositive statement means.

\item Construct an $n\times n$ matrix $A$ such that $\lambda_{\max}(A) \ne \sup\limits_{v\ne0}\big\{\frac{v'Av}{v'v}\big\}$, where $\lambda_{\max}(\cdot)$ denotes the maximum eigenvalue of its argument.  Why does your counter example not violate the Courant-Fischer theorem?

\end{enumerate}


\end{document}